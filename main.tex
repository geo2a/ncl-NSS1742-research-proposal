\documentclass[12pt, a4paper]{article}
\usepackage{url}

% correct bad hyphenation here

\title{Research Proposal}
\author{}
\date{}

\newcommand{\namelistlabel}[1]{\mbox{#1}\hfil}
\newenvironment{namelist}[1]{%1
\begin{list}{}
    {
        \let\makelabel\namelistlabel
        \settowidth{\labelwidth}{#1}
        \setlength{\leftmargin}{1.1\labelwidth}
    }
  }{%1
\end{list}}

\begin{document}
\maketitle

\begin{namelist}{xxxxxxxxxxxx}
\item[{\bf Title:}]
  Bridging Event-B and Workraft with statically typed functional embedded 
  domain-specific language.
\item[{\bf Project:}]
  Formal Methods for Design of Safety-Critical Processors (NSS1742)
\item[{\bf Author:}]
  Georgy Lukyanov
\item[{\bf Supervisors:}]
  Andrey Mokhov, Alexander Romanovsky
\end{namelist}

\section*{Background} 

\emph{Domain-Specific Languages} (DSLs) are designed to have a maximal 
expression for tasks in a particular domain (for example, VHDL for hardware
description or \LaTeX~for typesetting). However, implementing a language
from scratch may be tedious, time-consuming and error-prone. Therefore,
DSLs are often embedded into existing general-purpose programming languages,
which is particularly convenient for prototyping purposes.

Modern functional programming languages such as Haskell offer a wide range of
facilities for construction of \emph{Embedded Domain-Specific Languages} (EDSLs)
that benefit from features of lightweight formal verification provided bythe
rich type system and highly-tailored syntax achieved using various functional
programming idioms~\cite{HudakDSLs}.

To design resilient and reconfigurable systems, it is vital to have formal
specification methods, simulation facilities and verification techniques.
EDSLs can increase the productivity at every stage of hardware design:
high-level specification
languages help to describe the system functionality in a declarative way,
software simulation environments allow to evaluate the system capabilities
without fabricating an expensive prototype, and advanced types of the host
language provide compiler-checked correctness guarantees for synthesis.

The \textsc{Workcraft} framework provides three DSLs:

\begin{itemize}
\item Signal Transition Graphs (STGs), a signal-level DSL for
specifying resilient asynchronous controllers~\cite{STG}.
\item Conditional Partial Order Graphs (CPOGs), a DSL for specifying
reconfigurable processor microarchitectures supported by optimal
instruction encoding algorithms~\cite{ISA-formal}.
\item Dataflow Structures (DFSs), a dataflow-level DSL
for specifying dataflow computation graphs~\cite{DFS}.
\end{itemize}

\textsc{Workcraft} can synthesise and export models described in these DSLs into
Verilog, a low-level language for hardware description, supported
by conventional EDA tool chain.

\section*{Aim} 

In my work, I plan to focus on bridging Event-B~\cite{EventB}, 
a high-level formal notation for the specification of system requirements and reconfiguration, and DSLs provided by \textsc{Workcraft}. 

Dr. Andrey Mokhov has already developed a prototype of a bridging language, 
named \emph{Farfalle}, an intermediate-level DSL embedded in Haskell for the description of reconfigurable processor microarchitectures. My goal is to refine
its design and develop a production-ready DSL for processor specification 
with formal semantics and hardware compilation/generation tools  

\section*{Methodology}

My closest aim is to perform exploration phase: 

\begin{itemize} 
\item Refinement of~\emph{Farfalle} architecture.
\item Exploration of effectful computation typing techniques for EDSL
construction (e.g. algebraic effects and effects handlers, monad transformers).
\item Exploration of different hosting languages (e.g. Haskell, Idris and proof-assistants like Coq, Agda, etc.) and its features in context of hardware specification domain.
\end{itemize}

Then, when hosting language and EDSL architecture have been chosen, 
and prototype has been implemented, \textsc{Workcraft} and Event-B integration 
phase may be started. In order to complete is, following large-scale goals
must be accomplished:

\begin{itemize} 
\item Connect Event-B and designed intermediate EDSL with bidirectional code
generator, supplying proofs of translation soundness.
\item Implement a code generator connecting designed EDSL to \textsc{Workcraft}
internal representations.
\end{itemize}

\begin{thebibliography}{7}

\bibitem{ISA-formal}
A. Mokhov et al.
\emph{``Synthesis of processor instruction sets from high-level ISA specifications''}. IEEE Transaction on Computers 2014, vol. 63(6).

\bibitem{workcraft_web}
    \textsc{Workcraft} framework homepage: \url{http://www.workcraft.org/}.

\bibitem{rec-proc}
  A. Mokhov, M. Rykunov, D. Sokolov, A. Yakovlev.
  \emph{``Design of Processors with Reconfigurable Microarchitecture''}.
  J. Low Power Electronics Application, 2014, vol. 4(1), pp. 26-43.

\bibitem{HudakDSLs}
  P. Hudak.
  \emph{``Modular Domain Specific Languages and Tools''}.
  Proceedings of the International Conference on Software Reuse, 1998, p. 134.

\bibitem{STG}
J. Cortadella et al. \emph{``Logic synthesis for asynchronous controllers and interfaces''}, Springer, 2012.

\bibitem{DFS}
  D. Sokolov, I. Poliakov, A. Yakovlev. \emph{``Analysis of static data flow structures''}. Fundamenta Informaticae, vol. 88(4), pp. 581-610, 2008.

\bibitem{EventB}
  J-R. Abrial. \emph{Modeling in Event-B: system and software engineering}. Cambridge University Press, 2010.

\end{thebibliography}

\end{document}
