\documentclass[10pt, a4paper]{article}
\usepackage{url}
\usepackage[scale=0.75]{geometry} % Reduce document margins

% correct bad hyphenation here

\title{Research Proposal}
\author{}
\date{}

\usepackage [style = gost-numeric] {biblatex}
\addbibresource {biblio.bib}

\newcommand{\namelistlabel}[1]{\mbox{#1}\hfil}
\newenvironment{namelist}[1]{%1
\begin{list}{}
    {
        \let\makelabel\namelistlabel
        \settowidth{\labelwidth}{#1}
        \setlength{\leftmargin}{1.1\labelwidth}
    }
  }{%1
\end{list}}

\begin{document}
\maketitle

\begin{namelist}{xxxxxxxxxxxx}
\item[{\bf Title:}]
  Formal Methods for Design of Safety-Critical Processors (NSS1742)
\item[{\bf Author:}]
  Georgy Lukyanov
\item[{\bf Supervisors:}]
  Andrey Mokhov, Alexander Romanovsky, Patrick Degenaar, Shang-Wei Lin
\end{namelist}

\section{Background} 

Many resilient systems rely on runtime reconfigurability
to adapt to continuously changing environment without any
human intervention. For example, biomedical implants must
be able to operate autonomously within patients, adapting
to short-term and long-term changes, with required lifetimes
in the order of decades. Runtime reconfigurability can be
achieved both in hardware and software; the latter is less challenging 
to implement, for the reason that creation of new hardware is extremely 
expensive and challenging in comparison to software development. However, the
former is often unavoidable, because autonomy and resiliency requirements often
could not be met with general-purpose hardware, thus making necessary to
develop a custom solution.

In contrast with post-design testing, that doesn't provide full correctness
guarantees, formal methods provide a systematic approach for developing complex
systems in a correct-by-construction manner. One reason of advancement of formal
methods in comparison to testing is maximal possible elimination of human errors.
For instance, techniques known as model checking perform exhaustive search through
entire space of possible states of the system and evaluating system status in
every single state, completely excluding any possibility of non-specified
behaviour (presuming correctness of specification).

One class of formal methods is advanced type systems provided by modern
programming languages. Powerful yet lightweight formal verification techniques,
provided by this languages are based on famous Curry-Howard correspondence ---
a direct relationship between computer programs and mathematical proofs that
was discovered by the American mathematician Haskell Curry and logician William
Alvin Howard in late 1960s. A perfectly written Philip Wadler's paper, called
``Propositions as Types''~\cite{Wadler:2015:PT:2847579.2699407} contains great
set of historical notes alongside with
points of view on Curry-Howard correspondence from different fields of
mathematics and computer science. 
Curry-Howard correspondence is of great value for software developers because
it provides a possibility to formulate desired
properties of programs in terms of types and automatically acquire proofs of
correctness of those programs through type checking.

\subsection{Domain-specific languages for safety-critical design}

\emph{Domain-Specific Languages} (DSLs) are designed to have a maximal 
expression for tasks in a particular domain (for example, VHDL for hardware
description or \LaTeX~for typesetting). However, implementing a language
from scratch may be tedious, time-consuming and error-prone. Therefore,
DSLs are often embedded into existing general-purpose programming languages,
which is particularly convenient for prototyping purposes.

Modern functional programming languages such as Haskell offer a wide range of
facilities for construction of \emph{Embedded Domain-Specific Languages} (EDSLs)
that benefit from features of lightweight formal verification provided by the
rich type system and highly-tailored syntax achieved using various functional
programming idioms~\cite{Hudak:1998:MDS:551789.853532}.

To design resilient and reconfigurable systems, it is vital to have formal
specification methods, simulation facilities and verification techniques.
EDSLs can increase the productivity at every stage of hardware design:
high-level specification languages help to describe the system functionality
in a declarative way, software simulation environments allow to evaluate the
system capabilities without fabricating an expensive prototype, and advanced
types of the host language provide compiler-checked correctness guarantees for
synthesis.

\section{Aim} 

The goal of this project is to advance the theory and
engineering practice of custom processor design, with
particular focus on safety-critical and energy-constrained
applications, such as biomedical implants.

The developed software tools will be based on Rodin 
(\url{www.rodintools.org}), Event-B (\url{http://www.event-b.org/}) and
Workcraft (\url{www.workcraft.org}) frameworks developed at Newcastle
University, which are known and used worldwide. One of the project supervisors,
Prof Romanovsky is a world-leading expert in Event-B and Rodin.

In my work, I plan to focus on bridging Event-B, a high-level formal notation
for the specification of system requirements 
and reconfiguration, a part of the Rodin tool set, and \textsc{Workcraft}. 

Dr. Andrey Mokhov and his research team have already developed a prototype of a
bridging language, named \emph{Farfalle}, an intermediate-level embedded
domain-specific language for the description of reconfigurable processor
microarchitectures. My goal is to refine its design and develop a
production-ready DSL for processor specification enriched with formal semantics 
and hardware generation tools, with assistance and expertise from Prof 
Shang-Wei Lin research group from Nanyang Technological University
~\cite{Lin:2014:CSC:2962288.2962317}.  

I have already started to contribute to Dr. Mokhov's team on-going research
work and we have published a joint paper "Prototyping Resilient Processing Cores
in Workcraft" which was accepted at the 2nd International Workshop on Resiliency
in Embedded Electronic Systems that is a part of DATE'17 --- the top European
conference on Electronic Design Automation that will take place in Lausanne,
Switzerland, in March 2017. We are currently preparing an extended version of
the paper, which will be published by IEEE, and my section on domain-specific
languages plays the central role in the paper.

\section{Methodology}

My closest aim is to perform exploration phase: 

\begin{itemize} 
\item Exploration of different type systems and hosting languages 
(e.g. Haskell, Idris and proof-assistants like Coq, Agda, etc.) and its 
features in context of hardware specification domain.
\item Exploration of EDSL architectures.
\item Refinement of~\emph{Farfalle} architecture.
\end{itemize}

Then, when hosting language and EDSL architecture have been chosen, 
and prototype has been implemented, \textsc{Workcraft} and Event-B integration 
phase may be started. In order to complete is, following large-scale goals
must be accomplished:

\begin{itemize} 
\item Connect Event-B and designed intermediate EDSL with bidirectional code
generator, supplying proofs of translation soundness.
\item Implement a code generator connecting designed EDSL to \textsc{Workcraft}
internal representations.
\end{itemize}

\printbibliography

\end{document}
